\begin{ejer}
	Enunciar con detalle y demostrar que el índice de un punto respecto a un camino cerrado
	se conserva por giros, homotecias y traslaciones.
\end{ejer}

\begin{sol}
Veamos el giro de centro $z_0$ y ángulo $\theta$.
$$\varphi(z) = z_0 + (z-z_0)e^{i\theta}$$
La homotecia de centro $z_0$ y módulo $\lambda\in\mathbb{R}^{\ast}$
$$\varphi_{\lambda}(z) = z_0 + (z-z_0)\lambda$$
Y la traslación de vector $z_0$
$$\varphi_{z_0} (z) = z+z_0$$
Sea $\gamma$ un camino cerrado en $\Omega=\Omega^{\circ}$
y $z\in\Omega\backslash\gamma{\ast}$.
Queremos ver si  $$Ind_{\gamma}(w)=Ind_{\varphi\circ\gamma} (\varphi(w_0)) = \frac{1}{2\pi i} \int_{\varphi\circ\gamma} \frac{dz}{z-\varphi(w_0)}$$ 
Por el ejercicio $11$ de la relación $8$ deducimos que
$$\frac{1}{2\pi i} \int_{\varphi\circ\gamma} \frac{1}{z-\varphi(w_0)} dz = \frac{1}{2\pi i} \int_{\gamma} f(\varphi(w))\varphi'(w) dw = \frac{1}{2\pi i} \int_{\gamma} \frac{\varphi'(w)}{\varphi(w)-\varphi(w_0)} dw$$
$$\varphi(w)-\varphi(w_0) = (w-w_0) \xi \hspace{1cm}
z_0+(w-z_0)e^{i\theta}-(z_0+(w_0-z_0))e^{i\theta} \hspace{1cm}\varphi'(w) = \xi \in \{ e^{i\theta},\lambda,1 \}$$
De esa forma tenemos 
$$\frac{1}{2\pi i} \int_{\gamma} \frac{\varphi'(w)}{\varphi(w)-\varphi(w_0)} = \frac{1}{2\pi i} \int_{\gamma} \frac{dw}{w-w_0} = Ind_{\gamma}(w_0)$$
\end{sol}


\begin{ejer}
	Sea $\rho : [-\pi,\pi]\rightarrow\mathbb{R}^+$ una función de clase $C^1$, con $\rho(-\pi) = \rho(\pi)$ y sea $\sigma :[-\pi,\pi] \rightarrow\mathbb{C}$ el arco definido por
	$$ \sigma(t) = \rho(t)e^{it} \hspace{1cm} \forall t\in [-\pi,\pi] $$
	Calcular $Ind_{\sigma}(z)$ para todo $z\in\mathbb{C}\backslash\sigma{\ast}$.
\end{ejer}
\begin{sol}

Sea $\mathbb{C}\backslash\sigma^{\ast}$ tiene una única componente conexa no acotada $U$ y $\forall z\in U, Ind_{\sigma}(z)=0$
$$\Omega = \{ z\in\mathbb{C} : \exists t\in[-\pi,\pi], z=|z|e^{it}, |z|<\varphi(t) \}$$
Como $\varphi : [-\pi,\pi] \rightarrow \mathbb{R}^+$ es continua alcanza su mínimo absoluto que es positivo.

$0\in\Omega$ y $\Omega$ es estrellado respecto al cero
(dado $z\in\Omega$ $\lambda z + (1-\lambda)0 \in\Omega$ con $\lambda\in[0,1]$), por tanto $\Omega$ es conexo.
$$\Omega \cup \sigma^{\ast} \cup U = \mathbb{C}$$
Si $z\in\Omega$, como el índice es constante en cada componente conexa tenemos que $Ind_{\sigma}(z) = Ind_{\sigma}(0)$
$$Ind_{\sigma}(0) = \frac{1}{2\pi i} \int_{\sigma} \frac{1}{z-0}dz = \frac{1}{2\pi i} \int_{-\pi}^{\pi} \frac{1}{\sigma(t)} \sigma'(t)dt = \frac{1}{2\pi i} \int_{-\pi}^{\pi} \frac{\varphi'(t)+i\varphi(t)}{\varphi(t)} dt$$

\end{sol}


% Ejercicio 3: Demostración de que los arcos tienen logaritmo derivable

% Ejercicio 4:  C menos omega no tiene componentes conexas acotadas y omega es homologicamente conexo, por tanto hay un logaritmo...