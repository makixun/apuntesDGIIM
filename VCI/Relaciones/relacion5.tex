\begin{ejer} %(*)
	Sea $f:\mathbb{C}\rightarrow\mathbb{C}$ una función verificando que
	$$ f(z+w) = f(z)f(w) \hspace{1cm} \forall z,w\in\mathbb{C} $$
	Probar que, si $f$ es derivable en algún punto del plano, entonces $f$ es entera. Encontrar todas las funciones enteras que verifiquen la condición anterior. Dar un ejemplo de una función que verifique dicha condición y no sea entera.
\end{ejer}

\begin{sol}
Vemos los posibles valores de $f(0)$
$$f(0) = f(0+0) = f(0)^2 \implies f(0)= 0 \text{ ó }f(0) = 1$$
Si $f(0)=0$ la función es constante, si es $f(0)=1$ y suponiendo que $f$ es derivable en $\alpha\in\mathbb{C}$:

$\exists \lim_{h\rightarrow 0} \frac{f(\alpha+h)-f(\alpha)}{h}$
vemos que cumple la fórmula de adición 
$$f'(\alpha) = \lim_{h\rightarrow 0}\frac{f(\alpha+h)-f(\alpha)}{h} = \lim_{h\rightarrow 0}f(\alpha) \frac{f(h)-f(0)}{h} \Longleftrightarrow f \text{ es derivable en } 0$$ y eso se puede aplicar $\forall z\in\Omega$ de la siguiente forma:
$$ \lim_{h\rightarrow 0}\frac{f(y+h+0)-f(y+0)}{h} = \lim_{h\rightarrow 0} \frac{f(y)f(h+0)-f(y)f(0)}{h} = f(y)\lim_{h\rightarrow 0} \frac{f(h+0)-f(0)}{h} =f(y)f'(0)  $$

Ahora encontramos todas las funciones enteras que cumplan la condición del enunciado.

Sea $z\in\Omega$, $f(z)=0$, $f(z) = e^{wz} : w\in\mathbf{C}$

Sea $f$ tal que $f\in\mathcal{H}(\mathbb{C})$,
$ f'(z) = f(z) \lim_{h\rightarrow 0} \frac{f(h)-f(0)}{h}$
$$(f(z) e^{-wz})' = f'(z) e^{-wz} + (-w) +ze^{-wz} = f(z)we^{-wz} - f(z)we^{-wz} = 0$$
Por lo que las funciones son iguales salvo una constante.
En el punto $0$ las dos funciones valen lo mismo, por lo que la función es la constante $1$, ya que $f(0)=1$


El ejemplo de función entera que verifique dicha condición es $f(z) = e^{Re(z)}$
\end{sol}


\begin{ejer}
	Calcular la imagen por la función exponencial de una banda horizontal o vertical y del dominio cuya frontera es un rectángulo de lados paralelos a los ejes.
\end{ejer}

\begin{sol}

Podemos ver la exponencial como $e^z = e^{Re(z)} (\cos(Imz) + i\sin(Imz))$, sean $a,b\in\mathbb{R}$ con $a<b$.
Definimos las bandas horizontales y verticales:
$$B_H = \{ z\in\mathbb{C} : a\leq Rez \leq b \} \hspace{1cm} B_V = \{ z\in\mathbb{C} : a\leq Imz \leq b \}$$
Si $a\leq Rez \leq b$, $e^a \leq e^{Rez} \leq e^b$
la función se mueve por toda la circunferencia.%? unidad.

Cuando la parte imaginaria se puede mover donde quiera le das infinitas vueltas a la circunferencia unidad. Tenemos que
$exp(B_H)$ es la corona circular de centro $0$ y radios $e^a$ y $e^b$.

Tenemos que $\exp(B_V)$ es el sector del plano encerrado entre los ángulos $a$ y $b$.

En el caso del dominio cuya frontera es un rectángulo de lados paralelos a los ejes su imagen quedaría una corona restringida a los ángulos determinados por la parte imaginaria de dicho rectángulo.


\end{sol}


\begin{ejer}
	Dado $\theta\in]-\pi,\pi[$, estudiar la existencia del límite en $+\infty$ de la función $\varphi : \mathbb{R}^+\rightarrow\mathbb{C}$ definida por $\varphi(r) = exp(re^{i\theta})$ para todo $r\in\mathbb{R}^+$.
\end{ejer}



\begin{ejer}
Probar que si $\{z_n\}$ y $\{w_n\}$ son sucesiones de números complejos, con $z_n \not = 0$ para todo
$n\in \mathbb{N}$ y $\{ z_n \} \rightarrow 1$, entonces
$$ \{w_n(z_n-1)\} \rightarrow \lambda\in\mathbb{C} \implies \{ z_n^{w_n} \}\rightarrow e^{\lambda} $$
\end{ejer}

\begin{sol}

Como la función exponencial es continua
$\{ w_n (z_n-1) \} \rightarrow \lambda \implies \{ e^{w_n(z_n-1)} \} \rightarrow e^{\lambda}$
$$\lim \{ \log(z_n) \} = 0 \implies \lim \{ z_n-\log(z_n) \} = 1 \implies \lim \{ w_n(z_n-\log(z_n))-w_n \} = 0$$
$$\lim \left\{ z_n^{w_n} = e^{\log(z_n)-w_n} \right\}= \lim \left\{ e^{w_n (z_n-1)} \right\} \Longleftrightarrow \lim\left\{ \frac{e^{w_n(z_n-1)}}{e^{w_n (\log(z_n))}} = e^{w_n (z_n-\log(z_n))-w_n} \right\} = 1$$
Vemos que
$$z_n^{w_n} = e^{w_n \frac{\log(z_n)}{z_n-1} (z_n-1)} =  e^{w_n (z_n-1) \frac{\log(z_n)}{z_n-1}}$$
Y sabemos que $ \frac{\log(z_n)}{z_n-1} \rightarrow 1$ ya que
$\lim_{z\rightarrow 1} \frac{\log(z)-\log(1)}{z-1} = \log'(1) = 1/1$

\end{sol}




\begin{ejer}
	Estudiar la convergencia puntual, absoluta y uniforme de la serie de funciones $\sum_{n\geq 0} e^{-nz^2}$
\end{ejer}

\begin{sol}
La serie converge puntualmente si, y sólo si, $\left| \frac{1}{e^{z^2}} \right| <1$, además
$$\left| \frac{1}{e^{z^2}} \right| <1 \Longleftrightarrow 1<|e^{z^2}|  \Longleftrightarrow 0<Rez^2 = (Rez)^2-(Imz)^2 
\Longleftrightarrow |Rez| > |Imz|$$
donde en la última implicación hemos usado
$e^{z^2} = e^{Rez^2} e^{Imz^2}$
$$Rez^2 = (Rez)^2 - (Imz)^2 \hspace{1cm}Rez^2>0 \Longleftrightarrow |Rez| > |Imz|$$
Vemos ahora la convergencia uniforme, definimos
$$A = \{ z\in\mathbb{C} : (Rez)^2>(Imz)^2 \}$$
Si un conjunto $B\subset A$ y satisface que $\inf_{z\in B} [ (Rez)^2-(Imz)^2 ] > 0$, entonces hay convergencia uniforme en $B$.
\end{sol}



\begin{ejer}
	Probar que $a, b, c \in\mathbb{T}$ son vértices de un triángulo equilátero si, y sólo si, $a+b+c = 0$.
\end{ejer}

\begin{sol}
Para que los vértices en $\mathbb{T}$ formen un triángulo equilátero deben tener entre ellos una diferencia en el argumento de $2\pi/3$, de esa forma los vértices quedan:
$$\{ a,b,c \} = \{ e^{(\lambda - 2/3\pi)i}, e^{\lambda i}, e^{(\lambda + 2/3\pi)i} \}$$

$\Longrightarrow$
$$e^{2/3\pi i} (a+b+c) = e^{2/3\pi i} a + e^{2/3\pi i}b + e^{2/3\pi i}c = a+b+c \implies a+b+c=0$$

$\Longleftarrow$
$$a'=\frac{a}{a} = 1, b'=\frac{b}{a}, c'=\frac{c}{a} \hspace{1cm} b'=e^{\theta i} , \ \ c' = e^{\gamma i} = -b-1$$
$$a+b+c = 0 \implies a'+b'+c' = 0 \implies 1+b'+c' = 0 \implies c'=-b'-1 \text{ con }\gamma,\theta\in ]-\pi,\pi[$$
De lo que deducimos que 
$(-\cos(\theta)-1)-i\sin(\theta) = (\cos(\gamma)) + i(\sin(\gamma))$, además
$$-\sin(\theta) = \sin(\gamma) \implies \theta = -\gamma 
\hspace{1cm}-\cos(\theta) -1 =  \cos(\gamma)$$
por tanto
$$\theta = \pm\frac{2\pi}{3} = -\gamma$$
\end{sol}




\begin{ejer}
	Sea $\Omega$ un subconjunto abierto no vacío de $\mathbb{C}^{\ast}$ y $\varphi\in\mathcal{C}(\Omega)$ tal que $\varphi(z)^2 = z$ para todo $z\in\Omega$. Probar que $\varphi\in\mathcal{H}(\Omega)$ y calcular su derivada.
\end{ejer}

\begin{sol}
$a\in\Omega$ y vemos que es derivable por la definición
$$\lim_{z\rightarrow a} \frac{\phi(z)-\phi(a)}{z-a} \frac{\phi(z)+\phi(a)}{\phi(z)+\phi(a)} = \lim_{z\rightarrow a} \frac{z-a}{z-a} \frac{1}{\phi(z)+\phi(a)} = \frac{1}{2\phi(a)} = \phi '(a)$$
donde hemos usado que $\phi(a)\not = 0$.

\end{sol}



\begin{ejer}
	Probar que, para todo $z \in D(0, 1)$ se tiene:
	\begin{enumerate}[label=(\alph*)]
		\item $\sum_{n=1}{\infty} \frac{(-1)^{n+1}}{n} z^n = \log(1+z)$
		\item $\sum_{n=1}^{\infty} \frac{z^{2n+1}}{n(2n+1)} = 2z-(1+z)\log(1+z)+(1-z)\log(1-z)$
	\end{enumerate}
\end{ejer}

\begin{sol}

\textbf{a)}
$g = \log (1+z) \in\mathcal{H}(D(0,1))$ y $(\log (1+z))' = \frac{1}{1+z}$ $\forall z\in D(0,1)$

$\frac{1}{1+z} = \frac{1}{1-(-z)} = \sum_{n=0}^{\infty} (-1)^n z^n$
por otra parte la serie de potencias 

$\sum_{n\geq 1} \frac{(-1)^{n+1}}{n} z^n$ tiene radio de convergencia $1$
y su suma $\sum_{n=1}^{\infty} \frac{(-1)^{n+1}}{n} z^n = f(z)$ es holomorfa en $D(0,1)$ y su derivada se calcula término a término
$$f'(z) = \sum_{n=1}^{\infty} \frac{(-1)^{n+1}}{n} n z^{n-1} = \sum_{n=0}^{\infty} (-1)^{n}z^{n}$$
Entonces $f'(z) = g'(z)$ $\forall z\in D(0,1)$, por tanto $f$ y $g$ difieren en una constante.

Como $g(0) = \log(1) = 0 = f(0)$
con lo que tenemos que $f$ y $g$ son iguales en $D(0,1)$

\end{sol}




\begin{ejer}
	Probar que la función $f:\mathbb{C}\backslash\{1,-1\}\rightarrow\mathbb{C}$ definida por
	$$ f(z) = \log\left( \frac{1+z}{1-z} \right) \hspace{1cm} \forall z\in\mathbb{C}\backslash\{1,-1\} $$
	es holomorfa en el dominio $\Omega = \mathbb{C}\backslash\{ x\in\mathbb{R} : |x|\geq 1 \}$ y calcular su derivada. Probar también que
	$$ f(z) = 2\sum_{n=0}^{\infty} \frac{z^{2n+1}}{2n+1} \hspace{1cm} \forall z\in D(0,1) $$
\end{ejer}


\begin{sol}

$f(z) = \log(\frac{1+z}{1-z})$
la función es holomorfa en $\mathbb{C}\backslash \{1\}$
sabemos que $\log \in \mathcal{H}(\mathbb{C}^{\ast}\backslash\mathbb{R}^-)$
y vemos cuando la función logaritmo cae dentro de dicho conjunto

de forma intuitiva
$\frac{1+z}{1-z} \in\mathbb{R}^- \Longleftrightarrow \exists r>0 : z\not=1, \frac{1+z}{1-z} = -r \Longleftrightarrow 1+z = rz-r \Longleftrightarrow z(r-1)=1+r \Longleftrightarrow z = \frac{1+r}{r-1}$

Viendo que
$g(z) = \frac{1+z}{1-z} \in\mathcal{H}(\Omega)$ y $g(\Omega) \subseteq \mathbb{C}^{\ast}\backslash \mathbb{R}^-$
podemos asegurar que $f\in\mathcal{H}(\mathbb{C})$ por composición.
$$f'(z) = \frac{ (1-z+(1+z))/(1-z)^2 }{ (1+z)/(1-z) } = \frac{2}{1-z^2}$$
Siendo $\xi\in\mathcal{H}(\Omega)$, entonces
$\xi '(z) = \frac{1}{1+z} + \frac{1}{1-z} = \frac{2}{1-z^2}$



\textbf{Pista}
hacer $\frac{2}{1-z^2}$ en serie de potencias
\end{sol}





\begin{ejer}
	Sean $\alpha,\beta\in [-\pi,\pi]$ con $\alpha < \beta$ y consideremos el dominio $\Omega = \{ z\in\mathbb{C}^{\ast} : \alpha < arg(z) < \beta \}$. Dado $\rho\in\mathbb{R}^+$ tal que $\rho\alpha,\rho\beta\in[-\pi,\pi]$, probar que definiendo $f(z) = z^{\rho}$ para todo $z\in\Omega$, se obtiene una biyección de $\Omega$ sobre el doninio $\Omega_{\rho} = \{ z\in\mathbb{C}^{\ast} : \rho\alpha < arg(z) < \rho\beta \}$, tal que $f\in\mathcal{H}(\Omega)$ y $f^{-1}\in\mathcal{H}(\Omega_{\rho})$.
\end{ejer}
\textbf{Pista}
$z^{\phi} = e^{ \phi \log z}$ con $\Omega,\Omega_{\phi} \subset \mathbb{C}^{\ast}\backslash \mathbb{R^-}$
$$f^{-1}(z) = \phi^{\frac{1}{\phi\log z}} = z^{1/{phi}}$$


\begin{ejer}
	Estudiar la convergencia de la serie $\sum_{n\geq 0} \frac{\sin(nz)}{2^n}$
\end{ejer}
\textbf{Pista}

$\sin(nz) = \frac{e^{inz}-e^{-inz}}{2i}$, entonces
$\frac{1}{2i} \sum_{n\geq 0} \frac{e^{inz}-e^{-inz}}{2^n}$
Tenemos que ver cuando, para un $z\in\mathbb{C}$ fijo, estudiar la convergencia de las series
$\sum_{n\geq 0} \frac{e^{inz}}{2^n}$ y $\sum_{n\geq 0} \frac{e^{-inz}}{2^n}$

$e^{inz} = e^{in(Rez+iImz)} = e^{-nImz+inRez} = e^{-nImz}e^{inRez}$ con $|e^{inRez}|=1$

$|\frac{e^{inz}}{2^n}| = \frac{e^{-nImz}}{2^n} = (\frac{e^{-Imz}}{2})^n$
entonces

$\sum_{n\geq 0} |\frac{e^{inz}}{2^n}|$ converge $\Longleftrightarrow e^{-Imz} < 2 \Longleftrightarrow -Imz < \ln 2 \Longleftrightarrow -\ln 2 < Imz$

y tenemos convergencia uniforme en $B\subset A= \{ z\in\mathbb{C} : |Imz|<\ln 2 \}$ tal que $\sup_{z\in B} |Imz| < \ln 2$


\begin{ejer}
	Sea $\Omega = \mathbb{C}\backslash\{ x\in\mathbb{R} : |x|\geq 1 \}$. Probar que existe $f\in\mathcal{H}(\Omega)$ tal que $\cos(f(z)) = z$ para todo $z\in\Omega$ y $f(x)==\arccos(x)$ para todo $x\in]-1,1[$. Calcular la derivada de $f$.
\end{ejer}


\begin{ejer}
	Para $z\in D(0,1)$ con $Re(z) \not =0$, probar que
	$$ \arctan\left(\frac{1}{z}\right) + \sum_{n=0}^{\infty} \frac{(-1)^n}{2n+1}z^{2n+1} = \left\{ \begin{array}{lcc}
	\pi/2  &   si  & Re(z)>0 \\
	-\pi/2 &  si & Re(z)<0 \end{array}
	\right. $$
\end{ejer}